% !TeX spellcheck = en_US
% !TeX encoding = UTF-8
\chapter*{Abstract}
Convolutional Neural Networks are the state-of-the art architectures in the field of Deep learning. These non-linear networks are well known for their extra-ordinary performance in tasks like image classification. However, the reason for their efficient performance has remained in the dark side and in the recent years putting light in to the Black-box has become the interesting topic of research. This thesis focuses on visualizing the filters of a Convnet using existing methods. And then analyze the effect of different parameters on the pattern recognition of the filters. In order to show that addition of Batch-Normalization layer and dropout has an effect on feature extraction nature of filters two models were built, where one is a basic model and the other had regularization techniques. For training these models MNIST dataset was used. And for more insights the model was also trained using CIFAR-10 dataset and visualized the activations for the images in that dataset. The findings of this thesis also include that training the model for more number of epochs resulted in less or no dead-filters. 

\textit{Index words: Non-linear models, Visualizations, dead-filters, redundant filters, regularization techniques.}